% Basisformatierungen
\documentclass[12pt,oneside,pdftex,a4paper,openany,DIV=calc,bibliography=totocnumbered,listof=numbered]{scrbook}
\usepackage[inner=2.41cm,outer=2.41cm,top=2.41cm,bottom=2.41cm,includeheadfoot]{geometry}

% Zeilenabstand 1,5em
\usepackage[onehalfspacing]{setspace}

% Absatz Neue Zeile durch Leerzeile
\usepackage{parskip}

% Absatz Neue Zeile 1em
%\setlength{\parskip}{16pt}

% Absatz Neue Zeile
% keine Einrueckung nach rechts
\setlength{\parindent}{0in}

% Bilderumgebung
\usepackage{graphicx}

% Sprache - Neue Rechtschreibung 
\usepackage[ngerman]{babel}

% Umlaute
\usepackage[latin1] {inputenc}

% t1 Fonts und schoene Schrift
\usepackage{lmodern}
\usepackage[T1] {fontenc}

%Kopf- und Fu�zeile
\usepackage{fancyhdr}
\pagestyle{fancy}
\fancyhf{}

%Kopfzeile links bzw. innen
\fancyhead[LO,RE]{\nouppercase{\leftmark}}

%Kopfzeile rechts bzw. aussen
\fancyhead[RO,LE]{\thepage}

%\fancyhead[RO,RE]{\includegraphics[width=3cm]{bilder/logo.jpg}}

%Linie oben
\renewcommand{\headrulewidth}{0.5pt}

%Linie unten
\renewcommand{\footrulewidth}{0.5pt}

% �berschreibung der Plainzuweisung von Chapterseiten zu Fancy --> Kopfzeile
\def\chapterpagestyle{fancy}

% Schusterjungenverhinderung
\clubpenalty = 10000 % schliesst Schusterjungen aus
\widowpenalty = 10000 % schliesst Hurenkinder aus

% Blindtext
\usepackage{blindtext}

% fussnotenzitation

%\usepackage[%
%authorformat={italic},%
%annotatorformat=italic,
%titleformat=all,%
%titleformat={all,commasep},%
%bibformat={shorttitle},%
%see,
%commabeforerest,
%]{jurabib}


% F�r Ausrichtungen von Grafiken und Tabellen
\usepackage{float} 
\restylefloat{table} 

%Floats f�r Bilder Text
\usepackage{wrapfig}

%listings
\usepackage{listings}

%url schoener formatieren
\usepackage{url}

% Bilder nebeneinander anordnen
%\usepackage{caption}
\usepackage[format=hang,justification=centering,singlelinecheck=off]{caption}
\usepackage{subcaption}

%Anmerkungen Ende eindeutschen
\usepackage{endnotes}
\renewcommand{\notesname}{Anmerkungen}

%Fussnoten Nicht bei jedem Chapter resetten
\usepackage{chngcntr} 
\counterwithout{footnote}{chapter}


\begin{document}

\section*{Titel}

Evaluierung der Toolchain f�r SVC-basiertes HTTP-Videostreaming

\section*{Mitglieder}
Sharif Hanini\\
Denis Oczko\\
Martin Maurer

\section*{Kurzbeschreibung}
Bisher erfolgt die Auslieferung von Videocontent f�r Streamingplattformen meist statisch oder mit mehreren 
festen, vorgegeben Aufl�sungen und Bitraten. Dadurch ist eine feingranulare Anpassung an die technische Anbindung und 
Dekodierf�higkeiten der immer zahlreicher werdenden Endger�te nicht m�glich.

Mittlerweile ist mit SVC (Scalable Video Coding) eine Technologie verf�gbar, die die M�glichkeit zur 
dynamischen Anpassung des Videostroms an �bertragungsrate, Framerate, Bitrate, Aufl�sung und technische 
Dekodierf�higkeiten der Endger�te zul�sst, ohne das mehrere separate Videofiles erstellt werden m�ssen. 

SVC ist durch einen eingebetteten H264/AVC-Videostrom, den sogenannten Baselayer, r�ckw�rtskompatibel zu H264/AVC 
Videodekodern. Durch die Abw�rtskompatibilit�t von SVC zu regul�rem H264/AVC steht in der Theorie dem Einsatz im HTTP-Streaming 
nichts im Wege. Allerdings mangelt es bisher an geeignetten Encodern und Decodern. 

In unserem Projekt 
wollen wir daher anhand des SVC Referenzcodecs die einzelnen Teile des HTTP-Streamings, wie Encoden, Distribution 
oder Decoden f�r verschiedene Einsatzbereiche und unterschiedlichen Videocontent praktisch erproben und 
evaluieren, inwieweit der Einsatz f�r HTTP-Streamings bereits m�glich ist und wo es noch Hindernisse gibt.

\section*{Funktionsweise von Scalable Video Coding}

\begin{figure}[hbt] \centering \includegraphics[width=0.90\textwidth]{bilder/svc1.jpg}
	\caption*{Funktionsweise SVC mit eingebetteten Baselayer}
	\label{fig1}
\end{figure}

\section*{M�gliche Votragsthemen}

\end{document}
